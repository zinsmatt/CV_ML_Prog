\part{Deep Learning}


\section{Yolo}
Basic idea for the Yolo (v1, v2, v3) object detection algorithm.

\begin{algorithm}[H]
\DontPrintSemicolon
\KwInput{Input image}
\KwOutput{a set of bounding boxes with confidence and label}
 Divide the image in a $S\times S$ grid. \\
\For{each cell}
  { predict: \\
  - B boxes ($c_x, c_y, w, h$ for each anchor box) \\
  - B confidence values (the confidence should be equal to $Pr(Object) \times IoU$, if there is no object in the cell it should be 0 and if there is if should be equal to the IoU with its bbox) \\
  - C class conditional probabilities ($Pr(class_i|Object$) \\}
 The final tensor is $S\times S \times (B \times 5 + C)$ \\
 Threshold and apply non-maximum suppression
\caption{Yolo}
\end{algorithm}


\section{Mask R-CNN}
Basic idea for the Mask R-CNN object detection and segmentation

\begin{algorithm}[H]
\DontPrintSemicolon
\KwInput{Input image}
\KwOutput{a set of bounding boxes + label + mask}
 Apply the \textit{backbone} network to compute feature maps \\
 Apply the RPN (Region Proposal Network) that outputs candidate RoI (objectness scores + box coordinates) \\
 Apply Fast R-CNN on each RoI with an additional branch for the mask:\\
 \For{each RoI}
 {
  apply Fast R-CNN with an additional branch for the mask: \\
 - classification: a set of class scores  \\
 - bbox regression: per-class bbox offsets \\
 - mask prediction: one binary mask per class \tcp*{only the mask corresponding to the true class is used during training}
 }
 \caption{Mask R-CNN}
\end{algorithm}

