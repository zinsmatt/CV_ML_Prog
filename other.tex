\section{Other}


\subsection{Canny filter and hysteresis}

Hysteresis is defined by two thresholds: $\beta_1, \beta_2$ with $\beta_1 > \beta_2$.
\begin{itemize}
    \item All points $< \beta_2$ are rejected.
    \item All points $> \beta_1$ are kept.
    \item The points between $\beta_2$ and $\beta_1$ are kept if they form a long enough chain.
\end{itemize}

\section{Linear Algebra}
\subsection{Properties of symmetric matrices}
A symmetric matrix $S$ ($S = S^T$) has the following properties:
\begin{itemize}
    \item $S$ has \textbf{only real} eigenvalues.
    \item is called \textbf{positive semi-definite} if $x^TSx \leq 0, \forall x$
    \item is called \textbf{positive definite} if $x^TSx > 0, \forall x \neq 0$
    \item all its eigenvectors associated to different eigenvalues are \textbf{orthogonal}
    \item is positive (semi)-definite, if all eigenvalues are positive (non negative).
    \item if $S$ is positive semi-definite and if we look at how $S$ transform the points on the unit circle $|x|=1$. The first eigenvalue is the value of the largest deformation and the last eigenvalue is the value of the smallest deformation. (if we sort eigenvalues in decreasing order)
\end{itemize}

\subsection{Skew-symmetric matrices}

A \textbf{skew-symmetric} or \textbf{anti-symmetric} matrix is such that: $A^T = -A$. It has the following properties:

\begin{itemize}
    \item All eigenvalues are either 0 or purely imaginary
    \item It cannot be diagonalized as a diagonal matrix but it can be diagonalized as a block-diagonal matrix.
    $A = V S V^T$, where $V$ is an orthogonal matrix an $S$ has its diagonal formed by $2\times 2$ matrices.
    \item The previous properties implies that the rank of $S$ can only be even.
\end{itemize}

The $3\times 3$ skew-symmetric is often used.
\begin{equation}
    \hat{U} = \left( \begin{array}{ccc}
        0 & -u_3 & u_2 \\
        u_3 & 0 & -u_1 \\
        -u_2 & u_1 & 0 
    \end{array} \right)
    \in \mathbb{R}^{3\times 3}
\end{equation}

It has 3 parameters: $u_1, u_2, u_3$.
It is called \textit{hat operator} and is a linear operator from $\mathbb{R}^3$ to $\mathbb{R}^{3\times 3}$. Also, multiplying a vector by this matrix is similar to the cross-product.
\begin{equation}
    \hat{U} V = U \times V
\end{equation}

The matrix $\hat{U}$ has rank 2 and thus a null-space of dimension 1. This null-space is actually spanned by the vector $U = (u_1, u_2, u_3)T$.


