\part{Programming}

\section{Arithmetic}

\subsection{Divisibility}
\subsubsection{Divisible by 3?}
A number is divisible by 3 if and only if the sum of its digits is divisible by 3.

\subsubsection{Divisible by 20?}
A number is divisible by 20 if it ends in: 20, 40, 60, 80 or 00.

\subsection{Chinese Reminder Theorem}

\theoremstyle{definition}
\begin{definition}{Chinese Reminder Theorem}
Suppose
\begin{equation}
    \begin{split}
        x \equiv a_1  \ (\textrm{mod}\ n_1) \\
        x \equiv a_2  \ (\textrm{mod}\ n_2) \\
        x \equiv a_3  \ (\textrm{mod}\ n_3)
    \end{split}
\end{equation}
If $n_1, n_2, n_3$ are pairwise prime (i.e $gcd = 1$ for each pair), their exist a solution and this solution is unique modulo $n = n_1 n_2 n_3$.
\end{definition}

\paragraph{Example}
Determine if a number is divisible by 60. The Chinese Reminder theorem tells us that a number is divisible by 60 if and only if it is divisible by 3 and 20.
