\part{Programming}
\chapter{Algorithme}

\section{Arithmetic}

\subsection{Divisibility}
\subsubsection{Divisible by 3?}
A number is divisible by 3 if and only if the sum of its digits is divisible by 3.

\subsubsection{Divisible by 20?}
A number is divisible by 20 if it ends in: 20, 40, 60, 80 or 00.

\subsection{Chinese Reminder Theorem}

\theoremstyle{definition}
\begin{definition}{Chinese Reminder Theorem}
Suppose
\begin{equation}
    \begin{split}
        x \equiv a_1  \ (\textrm{mod}\ n_1) \\
        x \equiv a_2  \ (\textrm{mod}\ n_2) \\
        x \equiv a_3  \ (\textrm{mod}\ n_3)
    \end{split}
\end{equation}
If $n_1, n_2, n_3$ are pairwise prime (i.e $gcd = 1$ for each pair), their exist a solution and this solution is unique modulo $n = n_1 n_2 n_3$.
\end{definition}

\paragraph{Example}
Determine if a number is divisible by 60. The Chinese Reminder theorem tells us that a number is divisible by 60 if and only if it is divisible by 3 and 20.


\section{Combinatorics}
\subsection{Combinations}

If you have $n$ elements, the number of sets of $k$ distinct elements that you can form is
\begin{equation}
    \left( \begin{array}{c}
        n \\ k
    \end{array} \right)
     = \frac{n!}{k!(n-k)!}
\end{equation}

\subsection{Combinations with repetitions}

Suppose now that you can take several times the same element. Again, the order of the elements has no importance. The number of sets with repetitions that can be formed becomes
\begin{equation}
    \left(\begin{array}{c}
        n+k-1 \\ k
    \end{array} \right)
     = \frac{(n+k-1)!}{k!(n-1)!}
\end{equation}


\chapter{C++}

\section{Keywords}

\subsection{nodiscard}
\begin{lstlisting}
[[nodiscard]] int f();
struct [[nodiscard]] my_struct { };
\end{lstlisting}
If a function is declared \textbf{nodiscard} or if it returns an enum or class declared \textbf{nodiscard} by value, the compiler is encouraged to \textbf{raise a warning} if the returned value is not used.
