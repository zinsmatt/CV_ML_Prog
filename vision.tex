\part{Vision}


\section{Conics and quadrics}

\subsection{Ellipses}
The general quadratic equation of a conic is: $Ax^2 + Bxy + Cy^2 + Dx + Ey + F = 0 $

It can be written in the homogeneous form:
\begin{equation}
(x~~y~~1) \left[\begin{array}{ccc}
     A & B/2 & d/2  \\
     B/2 & C & E/2 \\
     D/2 & E/2 & F
\end{array}\right] 
\left( \begin{array}{c}
     x  \\
     y \\
     1
\end{array} \right)
= 0
\end{equation}
In 2D projective geometry, all conics are equivalent under a projective transformation.

An ellipse is a specific form of conic and has the equation:
\begin{equation}
\frac{x^2}{a^2} + \frac{y^2}{b^2} = 1
\end{equation}

\subsubsection{Euclidean matrix form}
\begin{equation}
    (X - C)^T A (X - C) = 1
\end{equation}
$C$ is the ellipse center and $A$ is a 2x2 symmetric matrix defining its shape. The eigen vectors of $A$ are the two axes and their corresponding eigen values give $1/a^2$ and $1/b^2$.

It is possible to extract the rotation part out of $A$:
\begin{equation}
(X - C)^T {}^{e}R_{w}^T \left[ \begin{array}{cc}
    1/a^2 & 0 \\
    0 & 1/b^2
\end{array}
\right] {}^{e}R_{w} (X - C) = 1
\end{equation}

\subsubsection{Homogeneous form}
\begin{equation}
    X^T T_{-C}^T
    {}^{e}R_{w}^T
    \left[ \begin{array}{ccc}
    1/a^2 & 0 & 0\\
    0 & 1/b^2 & 0\\
    0 & 0 & -1
    \end{array}\right]
    {}^{e}R_{w}
    T_{-C}
    X = 0
\end{equation}

where $T_{-C} = \left[\begin{array}{ccc}
    1&0&-C_x \\
    0&1&-C_y \\
    0&0&1
    \end{array}\right]$ is a translation to move the ellipse center to~(0, 0).

This gives a complete homogeneous equation:
\begin{equation}
    X^T C X = 0
\end{equation}
where $C$ is a 3x3 symmetric matrix defined up to a scale (6 elements for only 5 parameters because of the scale).

\subsubsection{Dual form}
It defines the ellipse by tangent lines instead of points. It satisfies:
\begin{equation}
    l^T C^* l = 0
\end{equation}

where $C^*$ is the adjointe matrix of $C$ and we have $C^* \sim C^{-1}$

\subsubsection{Decompose an ellipse}
This show how to decompose an ellipse into: half-axes, orientation and center.
Input: $C$ the matrix of the dual conic

\begin{enumerate}
\item Normalize by minus the last element $C/-C[2, 2]$ to force C[2, 2] = -1
\item $c = -C[:2, 2]$ is the center of the ellipse
\item Apply translation of minus the center to get the ellipse centered. $C_{centered} = T_{-c} C T_{-c}^T $
\item Enforce symmetry: $C_{centered} = 0.5 (C_{centered}+C_{centered}^T)$
\item Compute eigen decomposition $D, R = eig(C_{centered}[:2, :2])$
\item $a, b = sqrt(abs(D))$, $orientation = R$ and $center = c$
\end{enumerate}

\subsubsection{Fit an ellipse}
An ellipse have 5 degrees of freedom (3x3 symmetric matrix: 6 parameters -1 for the scale).

Each point gives one equation, so we need at least 5 points.
\begin{equation}
    \left[
    \begin{array}{cccccc}
    x_i^2 & x_i y_i & y_i^2 & x_i & y_i & 1  \\
    \vdots &\vdots &\vdots &\vdots &\vdots &\vdots 
    \end{array}
    \right]
    \left[\begin{array}{c}
        a \\ b \\ c \\ d \\ e \\ f
    \end{array} \right]
    = 0
\end{equation}

solve this homogeneous system $Ax=0$ with SVD. The last row of $Vt$ gives the best estimation.


\subsection{Ellipsoids}

Ellipsoids are quadrics with the following type of equation:
\begin{equation}
    \frac{x^2}{a^2} + \frac{y^2}{b^2} + \frac{z^2}{c^2} = 1
\end{equation}

\subsubsection{Euclidean matrix form}
Similar to the ellipse case, the general Euclidean form of an ellipsoid is:
\begin{equation}
    (X - C)^T A (X - C) = 1
\end{equation}
$C$ is the ellipse center and $A$ is a 3x3 symmetric matrix defining its shape. The eigen vectors of $A$ are the three axes and their corresponding eigen values give $1/a^2$, $1/b^2$ and $1/c^2$.

Again, it is possible to extract the rotation out of $A$:

\begin{equation}
(X - C)^T {}^{e}R_{w}^T \left[ \begin{array}{ccc}
    1/a^2 & 0 & 0 \\
    0 & 1/b^2 & 0 \\
    0 & 0 & 1/c^2
\end{array}
\right] {}^{e}R_{w} (X - C) = 1
\end{equation}


\subsubsection{Homogeneous form}
\begin{equation}
    X^T T_{-C}^T
    {}^{e}R_{w}^T
    \left[ \begin{array}{cccc}
    1/a^2 & 0 & 0 & 0\\
    0 & 1/b^2 & 0 & 0\\
    0 & 0 & 1/c^2 & 0\\
    0 & 0 & 0 & -1
    \end{array}\right]
    {}^{e}R_{w}
    T_{-C}
    X = 0
\end{equation}

where $T_{-C} = \left[\begin{array}{cccc}
    1&0&0&-C_x \\
    0&1&0&-C_y \\
    0&0&1&-C_z \\
    0&0&0&-1
    \end{array}\right]$ is a translation to move the ellipsoid center to~(0, 0, 0).

This gives a complete homogeneous equation:
\begin{equation}
    X^T Q X = 0
\end{equation}

where $Q$ is a 4x4 symmetric matrix defined up to a scale (10 elements for only 9 parameters because of the scale).

\subsubsection{Dual form}
It defines the ellipsoid by tangent planes instead of points. It satisfies:
\begin{equation}
    l^T Q^* l = 0
\end{equation}

where $Q^*$ is the adjointe matrix of $Q$ and we have $Q^* \sim Q^{-1}$


\subsubsection{Decompose an ellipsoid}
This show how to decompose an ellipsoid into: half-axes, orientation and center.
Input: $Q$ the matrix of the dual quadric
\begin{enumerate}
\item Normalize by minus the last element $Q/-Q[3, 3]$ to force Q[3, 3] = -1
\item $c = -Q[:3, 3]$ is the center of the ellipsoid
\item Apply translation of minus the center to get the ellipsoid centered. $Q_{centered} = T_{-c} Q T_{-c}^T $
\item Enforce symmetry: $Q_{centered} = 0.5 (Q_{centered}+Q_{centered}^T)$
\item Compute eigen decomposition $D, R = eig(Q_{centered}[:3, :3])$
\item $a, b, c = sqrt(abs(D))$, $orientation = R$ and $center = c$
\end{enumerate}

\section{2D Transformations}

\begin{tabular}{|c|c|c|c|}
 \hline
  Transformation & degrees of freedom & matrix & invariants \\
  \hline
  Projective & 8 dof & 
  $\left[\begin{array}{ccc}
       h_{11}&h_{12}&h_{13}  \\
       h_{21}&h_{22}&h_{23}  \\
       h_{31}&h_{32}&h_{33}  \\
    \end{array}\right]$ & 
    \begin{minipage}[t]{0.4\textwidth}
    \begin{itemize}
        \item lines (collinearity)  
        \item cross ratios (ratio of ratio)
    \end{itemize}
    \end{minipage}
    \\
    \hline
    
      Affine & 6 dof & 
  $\left[\begin{array}{ccc}
       a_{11}&a_{12}&t_{x}  \\
       a_{21}&a_{22}&t_{y}  \\
       0 & 0 & 1  \\
    \end{array}\right]$ & 
    \begin{minipage}[t]{0.4\textwidth}
    \begin{itemize}
        \item parallelism
        \item centroid
        \item line at infinity
    \end{itemize}
    \end{minipage}
    \\
    \hline
    
    Similarity & 4 dof & 
  $\left[\begin{array}{ccc}
       sr_{11}&sr_{12}&t_{x}  \\
       sr_{21}&sr_{22}&t_{y}  \\
       0 & 0 & 1  \\
    \end{array}\right]$ & 
    \begin{minipage}[t]{0.4\textwidth}
    \begin{itemize}
        \item angles
        \item ratio of length
        \item ratio of areas
    \end{itemize}
    \end{minipage}
    \\
    \hline
    
       Isometry (Euclidean) & 3 dof & 
  $\left[\begin{array}{ccc}
       r_{11}&r_{12}&t_{x}  \\
       r_{21}&r_{22}&t_{y}  \\
       0 & 0 & 1  \\
    \end{array}\right]$ & 
    \begin{minipage}[t]{0.4\textwidth}
    \begin{itemize}
        \item length
        \item angles
    \end{itemize}
    \end{minipage}
    \\
    \hline
\end{tabular}

\textbf{\underline{Note:}}

A similarity transform can be decomposed as a rotation $\theta$ and a non-uniform scaling $D$ with a certain orientation $\phi$:  $S = R(\theta)R(-\phi)DR(\phi)$

The group of invertible nxn matrices is the \textit{General Linear Group}.

The \textit{Projective Linear Group} is a subgroup of the matrices which are equivalent by a scalar factor.


\subsection{Topology of the projective plane}
Any point can be normalized such that: $x_1^2 + x_2^2 + x_3^2 = 1$.
Such point lies on the unit sphere $S^2$ in $R^3$
There is a 2-to-1 correspondence between $S^2$ and $P^2$.
Opposite points on the sphere represent the same point. Any two points on the sphere lies on a single great circle which correspond to the line between the two points. Any two great circles intersect into a single points (in fact 2 opposite points which represent the same point).

\subsubsection{Homogenous representation of points}
The homogeneous representation of a point is :
\begin{equation}
    \textbf{p} \sim \left(\begin{array}{c}
        \bar{p}\\1
    \end{array}\right)
    =
    \left(\begin{array}{c}
        x\\y\\1
    \end{array}\right)
\end{equation}

\textbf{P-normalization}
The mapping $R^2 \rightarrow P^2$ has an inverse mapping obtained by normalization.
\begin{equation}
    \textbf{x} = \left(\begin{array}{c}
        x_1 \\ x_2 \\ x_3
    \end{array}\right)
\end{equation}
and
\begin{equation}
    \textbf{x} \sim \left(\begin{array}{c}
        x_1 / x_3 \\ x_2 / x_3 \\ 1
    \end{array}\right)
\end{equation}


\subsubsection{Homogenous representation of lines}
Equation of a line in 2D: $l_1x + l_2y - \Delta = 0$

In homogeneous coordinates the line becomes:
\begin{equation}
    \textbf{l} \sim \left(\begin{array}{c}
        l_1 \\ l_2 \\ -\Delta
    \end{array}\right)
\end{equation}
In the canonical form, we assume $l_1^2 + l_2^2 = 1$ and $\Delta \ge 0$.
The previous equation becomes: $x^Tl=0$.


\textbf{D-normalization}
\begin{equation}
    \textbf{l} = \left(\begin{array}{c}
        \beta_1 \\ \beta_2 \\ \beta_3
    \end{array}\right)
\end{equation}

To go back to the canonical form, we do:
\begin{equation}
    \textbf{l}\sim \frac{-sign(\beta_3)}{\sqrt{\beta_1^2+\beta_2^2}} \left(\begin{array}{c}
        \beta_1 \\ \beta_2 \\ \beta_3
    \end{array}\right)
    = \left(\begin{array}{c}
        l_1 \\ l_2 \\ -\Delta
    \end{array}\right)
\end{equation}



\section{Gaussian and $\chi^2$ distributions}
\subsubsection{Gaussian probability distribution}
Given a vector $X$ of random variables $x_i$ for $i=1,\dots,N$, with mean $\bar{X} = E[X]$, and $\Delta_X = X - \bar{X}$, the covariance matrix $\Sigma$ is a NxN matrix given by

\begin{equation}
    \Sigma = E[\Delta_X \Delta_X^T]
\end{equation}

so that $\Sigma_{ij}=E[\Delta_{x_i}\Delta_{y_i}]$. The diagonal elements of $\Sigma$ are the variances of the individual variables $x_i$, and the off-diagonal elements are the cross-variances values.
In non-degenerate cases, where $\Sigma$ is positive-definite and thus invertible, the multivariate Gaussian distribution is
\begin{equation}
P(X)=\frac{1}{(2\pi)^{N/2}|\Sigma|^{1/2}} e^{-\frac{1}{2}(X-\bar{X})^T \Sigma^{-1} (X-\bar{X})}
\end{equation}

where $|\Sigma|$ is the determinant of $\Sigma$.

In the case where $\Sigma$ is just a diagonal matrix with a unique scalar term on the diagonal, $\Sigma=\sigma^2 I$, the distribution is called an \textit{isotropic Gaussian distribution}.

\subsubsection{Mahalanobis distance}

It is defined as:
\begin{equation}
    \norm{X - Y}_{Sigma}=\sqrt{(X-Y)^T \Sigma^{-1} (X-Y)}
\end{equation}{}

where $\Sigma$ is a positive definite matrix. This defines a metrix on $R^N$.
We can see that the Gaussian distribution is equivalent (without the normalization terms) to 

\begin{equation}
    P(X) \approx e^{-\frac{\norm{X-\bar{X}}_{\Sigma}^{2}}{2}}
\end{equation}{}

\subsubsection{Change of coordinates}
Since $\Sigma$ is symmetric and positive-definite, it may be rewritten as $\Sigma=U^T D U$, where $U$ is an orthogonal matrix and $D=(\sigma_1^2, \sigma_2^2, \dots, \sigma_N^2)$ is diagonal.

This shows that a general Gaussian PDF can be changed into an \textit{isotropic Gaussian distribution} with an orthogonal change of coordinates $X^\prime = UX$ and a scaling $\sigma_i$ of each component.
This also show that Euclidean distance and Mahalanobis distances are equivalent under a change of coordinates.

\subsection{$\chi^2$ distribution}

The $\chi_n^2$ distribution is the distribution of the sum of squares of $n$ independant Gaussian random variables. For a Gaussian random vector $v$, the value ${(v-\bar{v})^T \Sigma^{-1} (v-\bar{v})}$ satisfies a $\chi_n^2$ distribution, where $n$ is the dimension of $v$.
